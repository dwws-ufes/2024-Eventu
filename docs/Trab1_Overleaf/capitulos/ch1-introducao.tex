% ==============================================================================
% Projeto de Sistema - Nome do Aluno
% Capítulo 1 - Introdução
% ==============================================================================
\chapter{Introdução}
\label{sec-intro}
\vspace{-1cm}

Este documento apresenta o projeto (\textit{design}) do sistema \emph{\imprimirtitulo}. 
O projeto consiste no desenvolvimento de uma aplicação web para organizadores de eventos acadêmicos. Eventos acadêmicos possuem necessidades particulares que não são normalmente supridas por sistemas de vendas de ingressos, que geralmente tem foco em eventos culturais e shows. Por isso, entendemos como necessário desenvolver uma plataforma voltada especificamente para o público universitário.

Os principais diferenciais do sistema Eventu são o suporte a multiplas atividades paralelas, em que os participantes dispõem de diversas atividades para escolher e o sistema faz a gestão da grade de programação, e a API para controle de presença, em que a participação numa programação pode ser registrada com auxilio da integração a um sistema de catracas.

Este documento está organizado da seguinte forma: 
a Seção~\ref{sec-plataforma} apresenta a plataforma de software utilizada na implementação do sistema;
a Seção~\ref{sec-arquitetura} apresenta a arquitetura de software; por fim, 
a Seção~\ref{sec-frameweb} apresenta os modelos FrameWeb que descrevem os componentes da arquitetura.

O projeto foi desenvolvido por Thaliys Antunes Daré e Gabriel Ferrari Wagnitz.

